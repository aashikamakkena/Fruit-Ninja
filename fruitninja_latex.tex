\documentclass{beamer}

\usepackage{graphicx}
\usepackage{caption}

\usefonttheme{serif}
\usetheme{EastLansing}
\usecolortheme{orchid}


\title{\bfseries BVRIT Hyderabad College of Engineering for Women}
\subtitle{\bfseries FRUIT NINJA}
\author{\bfseries TEAM 13}
\begin{document}
 \begin{frame}
 \titlepage
  \end{frame}
  
  
 \begin{frame}
 \frametitle {\bfseries TEAM MEMBERS}
 \begin{itemize}
 \item S.Sriya Varma : 20WH1A0512 (CSE-A) 
 \item T.Praveenya : 20WH1A0477 (ECE-B)
 \item K.Varshini : 20WH1A0434 (ECE-A)
 \item P.Koushika : 20WH1A6617 (CSE-AIML)
 \item M. Sai Aashika : 20WH1A1203 (IT-A)
 \end{itemize}
 \end{frame} 
 
 \begin{frame}
\Huge
\frametitle{\bfseries FRUIT NINJA}
\bigskip
\begin{figure}
 \includegraphics[width = 10cm]{fruit ninja}
 \caption{GAME SCREEN}
\end{figure}
\end{frame}

\begin{frame}
\frametitle{\bfseries INTRODUCTION}
\bigskip

\begin{itemize}
 \item Fruit ninja game, also known as fruit-slicing game.
 \item Developed by {\bfseries Halfbrick}.
 \item Published on April 21, 2010.
 \item {\bfseries Low cost} and {\bfseries addictive gameplay}.

\end{itemize}
\end{frame}

\begin{frame}
\frametitle{\bfseries OBJECTIVE}
\bigskip
\begin{itemize} 
  \item Design a fruit ninja game using Python.\\ 
  \item Built with the help of pygame module and basic concept of Python.\\
  \item Steps Involved in the Game Play:
  
  \large 
   \begin{itemize}
   \item Slicing of fruits
   \item Avoid bombs
   \item Score maximum points
   \item Game terminates after three bomb encounters
   \end{itemize}
\end{itemize}
    
\end{frame}  

\begin{frame}
 \frametitle{\bfseries APPROACH}
 \begin{itemize}
 \item Understanding the given problem statement.
 \item Going through the pygame documentation.
 \item Following a sequence of steps.
\end{itemize}
\end{frame}



   
\begin{frame}
\frametitle{\bfseries PROJECT REQUIREMENTS}
\begin{itemize}
\item In this python project ,we require {\bfseries pygame} and {\bfseries random} modules of python.
\medskip
\item Install pygame and random by using the following:
\begin{itemize}
\item pip install pygame
\item pip install random
\end{itemize}
\end{itemize}
\end{frame}

\begin{frame}
\frametitle{\bfseries PROJECT IMPLEMENTATION}
 \begin{itemize}
 \item Importing required modules.
 \item Initializing window.
 \item Defining Functions.
 \item Game Loop.
 \end{itemize}
 \end{frame}
 

\begin{frame}
\frametitle{\bfseries CHALLENGES}

\begin{itemize}
\item Working in a team.
\item Latex software.
\item Game Code.
\item Debugging code errors.
\item GIT.
\end{itemize}
\end{frame}

\begin{frame}
\frametitle{\bfseries OVERCOMES}
\begin{itemize}
\item Coordinating with team members.
\item Working on daily basis.
\item Regular clarification of doubts.
\item Completing the allotted tasks on time.
\item Following reference materials , tutorials to build the code.
\end{itemize}
\end{frame}

\begin{frame}
\frametitle{\bfseries LEARNINGS}
\begin{itemize}
\item Team Work.
\item Basic usage of Latex software.
\item Pygame documentation for GUI.
\item Writing the piece of code.
\item Time Management.
\item Pushing contributions to GIT.
\end{itemize}
\end{frame}

\begin{frame}{\bfseries WORK CONTRIBUTION}
\centering
\begin{tabular}{|l|l|l|l|}
\hline
\bfseries NAME & \bfseries CON1 & \bfseries CON2 & \bfseries CON3 \\ \hline
Varshini & Latex code & Import libraries & Main Display \\ \hline
Praveenya & Latex code & Generation & -- \\ \hline
Sriya & Latex code & Random generation & Display Score \\ \hline
Koushika & Latex code & Slicing & Sounds \\ \hline
Aashiqa & Latex code & Lives & GameOver Screen \\  \hline

\end{tabular}

\begin{itemize}
\item CON : Contribution of the individual.
\end{itemize}
\end{frame}
 

 \begin{frame}
\Huge
\frametitle{\bfseries OUTPUT}
\bigskip
\begin{figure}
 \includegraphics[width = 10cm]{output ss} 
\end{figure}
\end{frame}

 \begin{frame}
\Huge
\frametitle{\bfseries GIT REPO}
\bigskip
\begin{figure}
 \includegraphics[width = 10cm]{git ss} 
\end{figure}
\end{frame}







 \begin{frame}
 \Huge\centerline{\bfseries THANK YOU}
 \end{frame}
 \end{document}